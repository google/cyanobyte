\documentclass[a4paper,12pt,oneside,pdflatex,italian,final,twocolumn]{article}

\usepackage[utf8]{inputenc}
\usepackage{parallel}
\usepackage{siunitx}
\usepackage{booktabs}
\usepackage{fancyhdr}

\usepackage[export]{adjustbox}
\usepackage[margin=0.5in]{geometry}
\addtolength{\topmargin}{0in}

\usepackage{libertine}
\renewcommand*\familydefault{\sfdefault}  %% Only if the base font of the document is to be sans serif
\usepackage[T1]{fontenc}

\title{ MCP4725 }
\author{ Nick Felker, Chris Frederickson }
\date{ 2019 }

\begin{document}

\pagestyle{fancy}

\lhead{ Nick Felker, Chris Frederickson }
\chead { 2019 }
\rhead{ MCP4725 v0.1.0 }


\onecolumn


\begin{figure}
\begin{minipage}{0.47\textwidth}

\section{Overview}
    Microchip 4725 Digital-to-Analog Converter
    \begin{itemize}
        \item Device address 98
        \item Address type 7-bit
        \item Little Endian
    \end{itemize}


\end{minipage}
\hfill

\end{figure}


\section{Register Description}
\begin{itemize}
\item EEPROM - If EEPROM is set, the saved voltage output will
be loaded from power-on.

\item Output voltage - VOut = (Vcc * value) / 4096
The output is a range between 0 and Vcc with
steps of Vcc/4096.
In a 3.3v system, each step is 800 microvolts.

\end{itemize}

\section{Technical specification}
\centering
\begin{tabular}{lcrr}
\toprule
 & Register Name & Register Address & Register Length \\
\midrule
EEPROM & EEPROM & 96 & 12 \\
VOut & Output voltage & 64 & 12 \\
\bottomrule
\end{tabular}

\raggedright

\section{Fields}



\raggedright

\subsection{Register EEPROM}
\centering
\begin{tabular}{lcr}
\toprule
  & Field Name & Bits \\
\midrule
digitalOut & Digital (binary) output &
12:0
\\
\bottomrule

\end{tabular}


\raggedright

\subsubsection{Field digitalOut }

Only allows you to send fully on or off


\begin{itemize}
\item GND (0) - Ground
\item VCC (4095) - Vcc (full power)
\end{itemize}




\raggedright

\section{Functions}

\centering
\begin{tabular}{lc}
\toprule
  & Description \\
\midrule
getVOut & get vout \\
setVOut & set vout \\
\bottomrule
\end{tabular}


\raggedright
\subsection{Function getVOut }
get vout \\

\centering
\begin{tabular}{lcr}
\toprule
  & Inputs & Return \\
\midrule
asVoltage &
value (float32)
, 
vcc (float32)

&
voltage
\\
\bottomrule
\end{tabular}



\raggedright
\subsection{Function setVOut }
set vout \\

\centering
\begin{tabular}{lcr}
\toprule
  & Inputs & Return \\
\midrule
asVoltage &
output (float32)
, 
vcc (float32)

&
void
\\
\bottomrule
\end{tabular}



\raggedright

\end{document}